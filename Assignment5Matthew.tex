\documentclass{article}
\usepackage{mathtools,amsmath,amssymb,amsthm,enumerate}
\usepackage[margin=1in]{geometry}

\usepackage{fancyhdr}
\pagestyle{fancy}
\fancyhf{}
\rhead{Chasity Savella, 
Matthew Popescu, 
Xuan Tang(900996708)
}
\lhead{\textbf{Assignment 5} - Spring CS311}
%\rfoot{Page \thepage}


%\title{\vspace{-3ex} \bf Assignment 5 \\[1ex]}
%\title{\vspace{-3ex} \bf Assignment 5 \\[1ex]\rm\normalsize CS 311, Spring 2015 \\ Due: June 3, 2014}

%\date{}
%\author{}

\begin{document}
%\maketitle
\paragraph{Problem 1} Use a reduction to show that the language $ALL_{TM}$ is undecidable
\[ALL_{TM} = \{ \langle M \rangle \;|\; \mbox{where $M$ is a TM and $L(M) = \Sigma*$} \}\]

\paragraph{} Lets suppose that a turing machine, $R$, decides $ALL_{TM}$. Using this assumption, lets show how to decide the acceptance problem, $A_{TM}$.  On the input $\langle M,w\rangle$, construct $M_1$ as follows:
\\\\
\noindent $M_1 = $ "On input string $x$, simulate $M$ on $w$: 
    \begin{enumerate}[\indent 1.]
    \item If $M$ accepts, \em accept.\em
    \item If $M$ rejects, \em reject."
    \end{enumerate}
\paragraph{} Note, that if $M$ accepts $w$, $L(M_1) = \Sigma*$. Otherwise, $L(M_1) = \emptyset$. This ensures that the TM we are about to construct will only deal with the empty language or the "all" language. Now, we can define TM $S$ as follows:
\\\\
\noindent $S = $ "On input $\langle M,w\rangle$, construct $M_1$ as shown above.
	\begin{enumerate}[\indent 1.]
	\item Run $R$ on input $\langle M_1\rangle$. If $R$ accepts, \em accept.\em
	\item Otherwise, \em reject."\em\\
	\end{enumerate}
\noindent Here we have shown that if we assume there exists some $R$ that can decide $ALL_{TM}$, we can construct a reduction that would make $A_{TM}$ decidable. Since $A_{TM}$ was proven undecidable (using diagonalization), there cannot exist an $R$ that decides $ALL_{TM}$. Therefore, it is undecidable.

\paragraph{Problem 2} A {\em useless state} in a Turing machine
is one that is never entered on any input string.  Consider the 
problem of determining whether a Turing Machine has any useless 
states.  Formulate this problem as a language and show that it 
is undecidable.

\paragraph{Solution} We should formulate the language as follows:
\[USELESS_{TM} = \{ \langle M,q \rangle \;|\; \mbox{where $M$ is a TM and $q$ is a state in $M$} \}\]
\noindent Suppose the turing machine $R$ decides USELESS. R takes a Turing machine and a state as input and determines if that state is useless. So, give R the input $\langle M, q_{accept}\rangle$ so it can determine if the accept state of M is useless. If it is, the language is empty. Lets define S as the turing machine that will decide $E_{TM}$, which was proven in the book to be undecidable. Give S the input M, and run R on $\langle M, q_{accept}\rangle$.
\\If R accepts, \em accept.\em
\\If R rejects, \em reject. \em
\\\\Therefore, using the assumption that USELESS is decidable, we found a reduction from $E_{TM}$ to USELESS showing that $E_{TM}$ is decidable. However, $E_{TM}$ was already proven to be undecidable, so our assumption is false and we can conclude that R does not exist.

\paragraph{Problem 3} If $A \leq_m B$ and $B$ is a regular language,
does this imply that $A$ is a regular language? Why or why not?
\paragraph{Solution} No it does not. The language $0^i1^i$ is not regular, but it can be mapped to the language, $0^i1^k$, which is regular.
\\\\The reduction function is as follows: $f(w) = 0^i1^i$ if $w \in A$ or $f(w) = 10$ if $w \notin A$.
\\\\This way, $w \in A$ iff $f(w) = 0^i1^i$ and $f(w) \in B$, where $i \geq 0$.
\\\\This shows that B is a regular language, but A is not.
\\The language $A$ is a Context Free Language, so it is also turing decidable (theorem 4.9). Therefore, we can conclude that $f$ is a computable function.

\paragraph{Problem 4} Prove that the language
\[ LOOP_{TM} = \{ \langle M \rangle \;|\; M \text{ is a TM and $M$ loops on all inputs} \} \]
is not recognizable.
\\
\paragraph{Solution} First, assume $LOOP_{TM}$ is recognizable. We will show this assumption to be false later.
\\$\overline{LOOP}_{TM}$ can be described as follows:
\[\overline{ LOOP}_{TM} = \{ \langle M \rangle \;|\; M \text{ is a TM and $M$ halts on at least one input} \} \]
\noindent We can build a recognizer for this turing machine by non-deterministically running every input until the TM halts for at least one of them. By definition, this makes it a recognizer.
\\\\ So far, we have assumed $LOOP_{TM}$ is turing recognizable and we have shown $\overline{LOOP}_{TM}$ is also turing recognizable. From this, we can conclude that both are decidable (theorem 4.22). 
\\\\ In order to prove our original assumption false, we must show a reduction from a currently known undecidable $TM$ to either $LOOP_{TM}$ or $\overline{LOOP}_{TM}$ to show that one of them is undecidable. We will show that $LOOP_{TM}$ is undecidable.
\\\\ Let $R$ be a decider for $LOOP_{TM}$. Now let $S$ be a decider for $A_{TM}$. Define it as follows:
\\\\
\noindent $S = $ "On input $\langle M, w\rangle$: 
    \begin{enumerate}[\indent 1.]
    \item Run $R$ on $M$.
    \item If $R$ accepts, accept.
    \item If $R$ rejects, reject."
    \end{enumerate}
\noindent Here we have shown that $A_{TM}$ is decidable using our assumption that $LOOP_{TM}$ is recognizable and therefore decidable. However, $A_{TM}$ was previously proven to be undecidable by diagonalization in the textbook. So, $LOOP_{TM}$ is undecidable, and therefore unrecognizable. 
\end{document}
